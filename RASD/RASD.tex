\documentclass[a4paper]{report}

\usepackage[dvipsnames]{xcolor}
\usepackage{listings}
\usepackage[utf8]{inputenc}
\usepackage{alloy-style}
\usepackage{textcomp}
\usepackage{amsmath}

% Package for pdf metadata
\usepackage{hyperref}
\usepackage{graphicx}
\usepackage{titlepic}

\usepackage[nottoc,notlof,notlot]{tocbibind} 
\renewcommand\bibname{References}

\hypersetup{
  pdftitle={RASD},
  pdfauthor={Carlo Dell'Acqua, Adriana Ferrari, Angelica Sofia Valeriani},                     % author
  pdfsubject={SafeStreet},
  pdfkeywords={RASD, SafeStreet, Software Engineering},
  colorlinks=true,
  linkcolor=blue,
  citecolor=black,
  filecolor=black,
  urlcolor=purple,
  linktoc=page
}

\setcounter{secnumdepth}{3}
  % Cover

\begin{titlepage}
        \titlepic{\includegraphics[width=9cm]{RASD_Images/logo.jpg}}
        \title{RASD}
        \author{Carlo Dell'Acqua, Adriana Ferrari, Angelica Sofia Valeriani}
\end{titlepage}

\begin{document}
\maketitle
  % Table of contents
  \tableofcontents

  % 1 - Introduction
  \chapter{Introduction}
  \section{External Interface Requirements}

\subsection{User Interface}
The following mock ups represent a basic idea of how the Mobile Application
and the Web Interface are supposed to look like.
Through the smart-phone application and web interface, users have access to complete SafeStreets functionalities, including the possibility of notifying violations.

\begin{figure}[H]
  \centering
  \includegraphics[width=\textwidth,height=.95\textheight,keepaspectratio]{RASD_Images/UserInterface/All.jpg}
  \caption{\textit{User interface mock ups}}
\end{figure}

\subsection{Hardware Interfaces}
Since the application runs over the Internet, the hardware of both the server and the client needs to be able to connect to the Internet.
\begin{itemize}
  \item Server-side: the hosting platform will provide all the necessary hardware interface.
  \item Client-side: e.g. Wi-Fi, 3G/4G. 
\end{itemize}
Client-side hardware should have a camera in order to have access to all functionalities of the application. It is also recommended, although not mandatory, that client-side hardware is provided with a GPS.


\subsection{Software Interface}
An API provided by SafeStreets allows third parties to access the read-only functionalities offered by the system. This way, third party companies and applications can have access and possibly embed in their software statistical data about accidents and street safety collected by SafeStreets. 
On the other hand, functionalities that involve the creation of new data are meant to be used only within SafeStreets official platforms and are not exposed to third parties.

\subsection{Communication Interface}
SafeStreets will communicate with third parties and with the client application using the standard HTTPS protocol, which guarantees encryption by default.
\section{Functional Requirements}

\subsection{R1: Allow a visitor to register on the mobile phone app or the web app}
\begin{itemize}
    \item The system must allow the visitor to provide credentials and personal data
    \item The system must verify the correspondence between the ID number provided by the visitor and their personal information
    \item The system must allow the visitor to verify the account with an e-mail or SMS verification code
    \item The system must send a recap of registration to email of the visitor
    \item The system must verify that there are no other registered users or authorities with the same e-mail or ID number
    \item The user to accept users data privacy conditions to successfully register to the system
\end{itemize}
\subsubsection{Scenario 1}
Mario Rossi saw the advertisement of SafeStreets and decided to download the mobile application in order to be allowed to notify violations and improve safety of streets in his city. After
opening the new app, he is asked to fill a form with all his personal information, full name, email address, mobile number, ID card, and phone number. To proceed with his
registration, after all having filled all the form, Mario clicks on the "Create account" button. The system verify the correspondence between inserted personal data and the ID card information,
and if the verification is positive, Mario is asked whether he prefers to confirm his registration by phone number or by email.
He chooses the alternative he prefers, then he immediately receives the verification code on the account he chose.
He can confirm his registration by clicking on the verification code. The system then sends a recap to notify the positive outcome of registration. Mario is now a Registered
User of SafeStreets; he can login to notify violations on the streets and is enabled to the use of all the functions for a basic user (not an authority) that the system provides.

\subsection{Allow a user to send pictures about street and parking violations}
\begin{itemize}
    \item A user can submit a picture to the system whenever they see a street violation
    \item SafeStreets analyses the image to read the license plait of the vehicle
    \item The user can manually attach the license plait to the image, to help the system with its validation
    \item The user must provide the type of violation, either selecting it from a list or manually typing it if it is not already in the system
    \item The user can help localizing the violation by inserting the name of the street where it occurred
\end{itemize}
\subsubsection{Scenario 2}
Mario Rossi is a user of SafeStreets, he has just finished working, and he wants to go back home, but he is prevented by a car, that is in double-row parking with his own. He waits for ten 
minutes to see if the owner of the car is coming, but then decide to notify the violation. He makes two pictures representing the violation, then logs into the mobile application, 
inserting his personal data, then he clicks on the button to notify a violation, he insert time (17.55), position ( Via Botticelli, Viale Romagna, 2, 20133 Milano MI), a brief description 
of the violation, and the two pictures. He is asked to click the category of the notified violation (Double-Row Parking), or to add it, in the field "other" if the type of violation is not still 
present in the system. He is then asked to insert the license plate of the vehicle involved in the violation in the mobile application, he does it, then the app verifies that it 
corresponds to the license plate identified with the photograph.

\subsection{Validation of reported violations}
\begin{itemize}
    \item Whenever a violation is reported by a user, it is sent to a random group of $k$ SafeStreets users
    \item Users who receive the notification about this approval process are then asked to approve or reject the report or declare neutral if they are not able to verify
    \item The report is validated if and only if the algebraic sum of approvals (+1), neutrality (0) and rejections (-1) is at least $v$
    \item Every user is supposed to respond, the system expects to receive k answers
    \item If the answers of the users for validation, after a settled timeout, are not all received, the answer is automatically considered neutral
\end{itemize}
\subsubsection{Scenario 3}
SafeStreets has received the notification of Mario Rossi regarding a violation of double-row parking. To verify the violation, it sends a request to a 
number k of users. The number of users is randomly chosen, and depend from the number of users logged in the application at the time the system received the report of Mario Rossi
and the users that were logged just a few time before, to increase probability they will be available to answer the report. K users receive the request for validation, and they start 
answering the report. The system receives reports and makes the algebraic sum of approvals and rejections. The calculated sum is equal to $m$ and comparing the values, the result of the 
inequation $m$>$v$ is positive; as a consequence, the system validates the report and registers the violation.

\subsection{Allow a user to see the areas where a violation is more likely to happen}
\begin{itemize}
    \item SafeStreets can aggregate data inputted by users to show relevant statistics about the frequency of street violations
    \item Users can see how many violations usually happen in their neighborhood, in their current location or in an area of their choice
    \item A user has access to the type and amount of violations that happened
\end{itemize}
\subsubsection{Scenario 4}
Mario Rossi searches for a language school to improve English for his daughter, he found different schools in his city on the internet, but he wants an area 
not too far from his house where the little girl can securely go also on foot. He has a list of five schools to decide. He is a user of SafeStreets, so 
for every school in his list, he looks on the web interface of the application for the statistics about car accidents in all the five zones. He insert the 
address of the school, and the system responds with a map, showing the near unsafe area if the school is in a safe area, or, if the school is in an unsafe 
area, showing more detailed data about frequency of car accidents and violations. This operation is repeated five times.

\subsection{Register the violation}
\begin{itemize}
    \item The reported violation must have been validated by the system
    \item The system checks the type of violation
    \item In case the violation is already supported by the system, the number of violations of that type is increased, the license plate of the involved vehicle is added to the list of offenders, and in a report that SafeStreets will deliver to authorities
    \item In case the violation is not supported by the system, the system updates the list of supported violations, then proceed as in the previous case
\end{itemize}
\subsubsection{Scenario 5}
Luca Bianchi is in his lunch break, with his colleagues, and he is waiting for crossing the road on his way to go back to the office. While stopped at the semaphore

\subsection{Allow authorities to see data about violations and who committed them}
\begin{itemize}
    \item Authorities can access all data about violations that a standard user can access
    \item Authorities also have access to more specific data about who committed a violation, like the license plait of the car or how many infractions have been associated to a specific car
\end{itemize}

\subsection{Allow any citizen to become a user by providing a document}
\begin{itemize}
    \item A citizen can register to SafeStreets by providing their ID number
    \item The system must allow a user to register with an email and a password, that will be asked every time they log in
\end{itemize}

\subsection{Allow authorities to register with a special user profile}
\begin{itemize}
    \item Authorities must first register as citizens
    \item A standard user profile can be upgraded to authority profile if it is verified by the system
    \item To obtain privileged access, the user must provide a valid document that proves their authority status
\end{itemize}

\subsection{Unsafe Areas Identification}
\begin{itemize}
  \item SafeStreets can cross information from different sources to identify potentially unsafe areas
  \item SafeStreets may be integrated with a public service, offered by the municipality, that provides such information
  \item Once identified, SafeStreets can suggest possible interventions to prevent accidents
\end{itemize}

\subsection{Traffic Ticket Generation}
\begin{itemize}
  \item SafeStreets can generate tickets for traffic violations reported from registered users
  \item The information is kept safe and intact through the chain of custody, from the end user to the local police officer issuing the tickets
\end{itemize}
\section{Performance Requirements}

\subsection{Reliability}
\subsubsection{Index MTBF}
\begin{itemize}
    \item 90\% of all response time should be less than 4.5 second
    \item The response times will be measured using HP LoadRunner (or similar tool) located behind the firewall and in front of the web servers
    \item Backend response times will be measured using the application server log files
    \item The system is provided to serve a huge number of users in parallel
    \item At the beginning the system is supposed to be correctly and quickly responding to 100'000 users
    \item As number of users is expected to be grown linearly, the system is provided of a flexible structure, to adapt to further number of requests
    \item Daily tests to check performance will be executed to guarantee the service
    \item The performance tests should not exceed 50\% CPU utilization during the execution time
\end{itemize}
\section{Design Constraints}

\subsection{Standards compliance}
  \begin{itemize}
    \item To guarantee the compatibility with the potential municipality accident information service, our backend software should communicate and be able to read data in a portable format like JSON or XML
    \item To transfer this data, the software should expose an HTTP REST API
  \end{itemize}

\subsection{Hardware limitations}
  \begin{itemize}
    \item The frontend application should be as lightweight as possible to support the diversity of mobile device hardware
    \item Devices with a camera resolution of less than 2MP should be marked as incompatible with SafeStreets due to the poor quality of the pictures that are taken with them
  \end{itemize}

\subsection{Other constraints}
  \begin{itemize}
    \item The language used to develop the backend application will be chosen from the ones that are most supported by the main cloud infrastructure providers, in order to have a wider choice of hosting plans
  \end{itemize}
\section{Software System Attributes}

\subsection{Reliability}
\subsubsection{Index MTBF}
\begin{itemize}
    \item Index MTBF must consider as failures those out of design conditions which place the system out of service, for example by an overload of user’s requests
    \item MTBF is of 600'000 hours
\end{itemize}

\subsubsection{Index MTTR}
\begin{itemize}
    \item Index MTTR must consider time of testing and eventual solution of bugs
    \item MTTR must be of two hours
\end{itemize}

\subsubsection{The system must verify position in an accurate way}
\begin{itemize}
    \item The system expects to receive two pictures			
    \item The received pictures must be of at least 2 Mp
    \item The first picture must represent a general overview of the violation, to identify street and area
    \item The second picture must allow the system to identify correctly the license plate, if the violation includes a car
    \item The second picture is simply an additive detail, if the violation doesn't include a car
    \item In the second picture, if the violation includes a car, the license plate must occupy at least a quarter of the total picture
    \item The system must verify the position provided by the user that sent a picture notifying a violation by comparison to the data stored and the GPS service
    \item The identification of coordinates must not be less precise than 3 metres in case of parking violation
\end{itemize}

\subsubsection{Availability}
\begin{itemize}
    \item The system must guarantee a 24/7 service
    \item The system must be available 99.99\%
    \item The system must be available when performing the standard routine tests for maintenance
    \item The system can bear slowdowns of the service in case of extraordinary maintenance due to overloading of requests
\end{itemize}

\subsection{Security}
\subsubsection{Secrecy of data received and of users information}
\begin{itemize}
    \item The system should never allow both non-registered users and registered users to see the identity of the person that notifies the violation
    \item User credentials, ID and password, have to be encrypted and safely stored
    \item Every user has to register with username and password, but the system provides for each user a multi-factor authentication
\end{itemize}

\subsubsection{Integrity of data}
\begin{itemize}
    \item The information sent by users (picture, date, time and position of the violation) has to be encrypted in order to keep the integrity of message, and to avoid manipulation of data
\end{itemize}

\subsubsection{Measures of security according to danger level}
\begin{itemize}
  \item Public statistics to identify the most dangerous areas are provided for SafeStreets
  \item SafeStreets makes a ranking of dangerousness (minimum, medium, high)
  \item The authorities can decide to improve security by highlighting streets in areas at minimum risk
  \item The authorities can decide to improve security by adding cameras in the areas at medium risk 
  \item The authorities can decide to improve security by both adding cameras and doubling patrol shifts  in the areas at large risk
\end{itemize}

\subsection{Maintainability}
\subsubsection{Testing overview}
\begin{itemize}
  \item The system is checked in its correct functioning by software testing
  \item Unit tests are performed every month in standard conditions
  \item Integration tests are performed every two weeks in standard conditions
  \item System tests are performed every week in standard conditions, in order to guarantee performances and by regression to solve eventual bugs due to software development
\end{itemize}

\subsubsection{The system is controlled and monitored}
\begin{itemize}
  \item The system is equipped with condition-monitoring algorithms 
  \item The condition-monitoring algorithms identify the functions that cause alarm
  \item A unit test is made as soon as a function is identified as potentially at risk
  \item An integration test is performed after the unit test of the potentially at risk function
  \item A system test is performed after the integration and unit test of the potentially at risk function
\end{itemize}

\subsubsection{The code must be clean and easy to understand}
\begin{itemize}
  \item Code must have a clear structure
  \item It is advisable to following the design patterns to provide a standard terminology and to make the software more adaptable to future extensions and improvements
  \item Complete and detailed documentation from a unit to the superior levels is mandatory in order to keep the maintainability on the highest level on the whole system
\end{itemize}

\subsection{Portability}
\begin{itemize}
  \item The software shall run on Windows, IOS, and Linux
  \item The software must be developed in Java
  \item The software can be developed using the standard API’s that span different types of operating systems
  \item The software will be transferred with no modifications to support the environment on another destination machine
  \item Binary translation must be supported by the system as adaptability is significant and practical for a program in “binary” (executable) form
  \item The architecture must be flexible
\end{itemize}


  % 2 - Overall description
  \chapter{Overall description}
  \section{External Interface Requirements}

\subsection{User Interface}
The following mock ups represent a basic idea of how the Mobile Application
and the Web Interface are supposed to look like.
Through the smart-phone application and web interface, users have access to complete SafeStreets functionalities, including the possibility of notifying violations.

\begin{figure}[H]
  \centering
  \includegraphics[width=\textwidth,height=.95\textheight,keepaspectratio]{RASD_Images/UserInterface/All.jpg}
  \caption{\textit{User interface mock ups}}
\end{figure}

\subsection{Hardware Interfaces}
Since the application runs over the Internet, the hardware of both the server and the client needs to be able to connect to the Internet.
\begin{itemize}
  \item Server-side: the hosting platform will provide all the necessary hardware interface.
  \item Client-side: e.g. Wi-Fi, 3G/4G. 
\end{itemize}
Client-side hardware should have a camera in order to have access to all functionalities of the application. It is also recommended, although not mandatory, that client-side hardware is provided with a GPS.


\subsection{Software Interface}
An API provided by SafeStreets allows third parties to access the read-only functionalities offered by the system. This way, third party companies and applications can have access and possibly embed in their software statistical data about accidents and street safety collected by SafeStreets. 
On the other hand, functionalities that involve the creation of new data are meant to be used only within SafeStreets official platforms and are not exposed to third parties.

\subsection{Communication Interface}
SafeStreets will communicate with third parties and with the client application using the standard HTTPS protocol, which guarantees encryption by default.
\section{Functional Requirements}

\subsection{R1: Allow a visitor to register on the mobile phone app or the web app}
\begin{itemize}
    \item The system must allow the visitor to provide credentials and personal data
    \item The system must verify the correspondence between the ID number provided by the visitor and their personal information
    \item The system must allow the visitor to verify the account with an e-mail or SMS verification code
    \item The system must send a recap of registration to email of the visitor
    \item The system must verify that there are no other registered users or authorities with the same e-mail or ID number
    \item The user to accept users data privacy conditions to successfully register to the system
\end{itemize}
\subsubsection{Scenario 1}
Mario Rossi saw the advertisement of SafeStreets and decided to download the mobile application in order to be allowed to notify violations and improve safety of streets in his city. After
opening the new app, he is asked to fill a form with all his personal information, full name, email address, mobile number, ID card, and phone number. To proceed with his
registration, after all having filled all the form, Mario clicks on the "Create account" button. The system verify the correspondence between inserted personal data and the ID card information,
and if the verification is positive, Mario is asked whether he prefers to confirm his registration by phone number or by email.
He chooses the alternative he prefers, then he immediately receives the verification code on the account he chose.
He can confirm his registration by clicking on the verification code. The system then sends a recap to notify the positive outcome of registration. Mario is now a Registered
User of SafeStreets; he can login to notify violations on the streets and is enabled to the use of all the functions for a basic user (not an authority) that the system provides.

\subsection{Allow a user to send pictures about street and parking violations}
\begin{itemize}
    \item A user can submit a picture to the system whenever they see a street violation
    \item SafeStreets analyses the image to read the license plait of the vehicle
    \item The user can manually attach the license plait to the image, to help the system with its validation
    \item The user must provide the type of violation, either selecting it from a list or manually typing it if it is not already in the system
    \item The user can help localizing the violation by inserting the name of the street where it occurred
\end{itemize}
\subsubsection{Scenario 2}
Mario Rossi is a user of SafeStreets, he has just finished working, and he wants to go back home, but he is prevented by a car, that is in double-row parking with his own. He waits for ten 
minutes to see if the owner of the car is coming, but then decide to notify the violation. He makes two pictures representing the violation, then logs into the mobile application, 
inserting his personal data, then he clicks on the button to notify a violation, he insert time (17.55), position ( Via Botticelli, Viale Romagna, 2, 20133 Milano MI), a brief description 
of the violation, and the two pictures. He is asked to click the category of the notified violation (Double-Row Parking), or to add it, in the field "other" if the type of violation is not still 
present in the system. He is then asked to insert the license plate of the vehicle involved in the violation in the mobile application, he does it, then the app verifies that it 
corresponds to the license plate identified with the photograph.

\subsection{Validation of reported violations}
\begin{itemize}
    \item Whenever a violation is reported by a user, it is sent to a random group of $k$ SafeStreets users
    \item Users who receive the notification about this approval process are then asked to approve or reject the report or declare neutral if they are not able to verify
    \item The report is validated if and only if the algebraic sum of approvals (+1), neutrality (0) and rejections (-1) is at least $v$
    \item Every user is supposed to respond, the system expects to receive k answers
    \item If the answers of the users for validation, after a settled timeout, are not all received, the answer is automatically considered neutral
\end{itemize}
\subsubsection{Scenario 3}
SafeStreets has received the notification of Mario Rossi regarding a violation of double-row parking. To verify the violation, it sends a request to a 
number k of users. The number of users is randomly chosen, and depend from the number of users logged in the application at the time the system received the report of Mario Rossi
and the users that were logged just a few time before, to increase probability they will be available to answer the report. K users receive the request for validation, and they start 
answering the report. The system receives reports and makes the algebraic sum of approvals and rejections. The calculated sum is equal to $m$ and comparing the values, the result of the 
inequation $m$>$v$ is positive; as a consequence, the system validates the report and registers the violation.

\subsection{Allow a user to see the areas where a violation is more likely to happen}
\begin{itemize}
    \item SafeStreets can aggregate data inputted by users to show relevant statistics about the frequency of street violations
    \item Users can see how many violations usually happen in their neighborhood, in their current location or in an area of their choice
    \item A user has access to the type and amount of violations that happened
\end{itemize}
\subsubsection{Scenario 4}
Mario Rossi searches for a language school to improve English for his daughter, he found different schools in his city on the internet, but he wants an area 
not too far from his house where the little girl can securely go also on foot. He has a list of five schools to decide. He is a user of SafeStreets, so 
for every school in his list, he looks on the web interface of the application for the statistics about car accidents in all the five zones. He insert the 
address of the school, and the system responds with a map, showing the near unsafe area if the school is in a safe area, or, if the school is in an unsafe 
area, showing more detailed data about frequency of car accidents and violations. This operation is repeated five times.

\subsection{Register the violation}
\begin{itemize}
    \item The reported violation must have been validated by the system
    \item The system checks the type of violation
    \item In case the violation is already supported by the system, the number of violations of that type is increased, the license plate of the involved vehicle is added to the list of offenders, and in a report that SafeStreets will deliver to authorities
    \item In case the violation is not supported by the system, the system updates the list of supported violations, then proceed as in the previous case
\end{itemize}
\subsubsection{Scenario 5}
Luca Bianchi is in his lunch break, with his colleagues, and he is waiting for crossing the road on his way to go back to the office. While stopped at the semaphore

\subsection{Allow authorities to see data about violations and who committed them}
\begin{itemize}
    \item Authorities can access all data about violations that a standard user can access
    \item Authorities also have access to more specific data about who committed a violation, like the license plait of the car or how many infractions have been associated to a specific car
\end{itemize}

\subsection{Allow any citizen to become a user by providing a document}
\begin{itemize}
    \item A citizen can register to SafeStreets by providing their ID number
    \item The system must allow a user to register with an email and a password, that will be asked every time they log in
\end{itemize}

\subsection{Allow authorities to register with a special user profile}
\begin{itemize}
    \item Authorities must first register as citizens
    \item A standard user profile can be upgraded to authority profile if it is verified by the system
    \item To obtain privileged access, the user must provide a valid document that proves their authority status
\end{itemize}

\subsection{Unsafe Areas Identification}
\begin{itemize}
  \item SafeStreets can cross information from different sources to identify potentially unsafe areas
  \item SafeStreets may be integrated with a public service, offered by the municipality, that provides such information
  \item Once identified, SafeStreets can suggest possible interventions to prevent accidents
\end{itemize}

\subsection{Traffic Ticket Generation}
\begin{itemize}
  \item SafeStreets can generate tickets for traffic violations reported from registered users
  \item The information is kept safe and intact through the chain of custody, from the end user to the local police officer issuing the tickets
\end{itemize}
\section{Performance Requirements}

\subsection{Reliability}
\subsubsection{Index MTBF}
\begin{itemize}
    \item 90\% of all response time should be less than 4.5 second
    \item The response times will be measured using HP LoadRunner (or similar tool) located behind the firewall and in front of the web servers
    \item Backend response times will be measured using the application server log files
    \item The system is provided to serve a huge number of users in parallel
    \item At the beginning the system is supposed to be correctly and quickly responding to 100'000 users
    \item As number of users is expected to be grown linearly, the system is provided of a flexible structure, to adapt to further number of requests
    \item Daily tests to check performance will be executed to guarantee the service
    \item The performance tests should not exceed 50\% CPU utilization during the execution time
\end{itemize}
\section{Design Constraints}

\subsection{Standards compliance}
  \begin{itemize}
    \item To guarantee the compatibility with the potential municipality accident information service, our backend software should communicate and be able to read data in a portable format like JSON or XML
    \item To transfer this data, the software should expose an HTTP REST API
  \end{itemize}

\subsection{Hardware limitations}
  \begin{itemize}
    \item The frontend application should be as lightweight as possible to support the diversity of mobile device hardware
    \item Devices with a camera resolution of less than 2MP should be marked as incompatible with SafeStreets due to the poor quality of the pictures that are taken with them
  \end{itemize}

\subsection{Other constraints}
  \begin{itemize}
    \item The language used to develop the backend application will be chosen from the ones that are most supported by the main cloud infrastructure providers, in order to have a wider choice of hosting plans
  \end{itemize}
\section{Software System Attributes}

\subsection{Reliability}
\subsubsection{Index MTBF}
\begin{itemize}
    \item Index MTBF must consider as failures those out of design conditions which place the system out of service, for example by an overload of user’s requests
    \item MTBF is of 600'000 hours
\end{itemize}

\subsubsection{Index MTTR}
\begin{itemize}
    \item Index MTTR must consider time of testing and eventual solution of bugs
    \item MTTR must be of two hours
\end{itemize}

\subsubsection{The system must verify position in an accurate way}
\begin{itemize}
    \item The system expects to receive two pictures			
    \item The received pictures must be of at least 2 Mp
    \item The first picture must represent a general overview of the violation, to identify street and area
    \item The second picture must allow the system to identify correctly the license plate, if the violation includes a car
    \item The second picture is simply an additive detail, if the violation doesn't include a car
    \item In the second picture, if the violation includes a car, the license plate must occupy at least a quarter of the total picture
    \item The system must verify the position provided by the user that sent a picture notifying a violation by comparison to the data stored and the GPS service
    \item The identification of coordinates must not be less precise than 3 metres in case of parking violation
\end{itemize}

\subsubsection{Availability}
\begin{itemize}
    \item The system must guarantee a 24/7 service
    \item The system must be available 99.99\%
    \item The system must be available when performing the standard routine tests for maintenance
    \item The system can bear slowdowns of the service in case of extraordinary maintenance due to overloading of requests
\end{itemize}

\subsection{Security}
\subsubsection{Secrecy of data received and of users information}
\begin{itemize}
    \item The system should never allow both non-registered users and registered users to see the identity of the person that notifies the violation
    \item User credentials, ID and password, have to be encrypted and safely stored
    \item Every user has to register with username and password, but the system provides for each user a multi-factor authentication
\end{itemize}

\subsubsection{Integrity of data}
\begin{itemize}
    \item The information sent by users (picture, date, time and position of the violation) has to be encrypted in order to keep the integrity of message, and to avoid manipulation of data
\end{itemize}

\subsubsection{Measures of security according to danger level}
\begin{itemize}
  \item Public statistics to identify the most dangerous areas are provided for SafeStreets
  \item SafeStreets makes a ranking of dangerousness (minimum, medium, high)
  \item The authorities can decide to improve security by highlighting streets in areas at minimum risk
  \item The authorities can decide to improve security by adding cameras in the areas at medium risk 
  \item The authorities can decide to improve security by both adding cameras and doubling patrol shifts  in the areas at large risk
\end{itemize}

\subsection{Maintainability}
\subsubsection{Testing overview}
\begin{itemize}
  \item The system is checked in its correct functioning by software testing
  \item Unit tests are performed every month in standard conditions
  \item Integration tests are performed every two weeks in standard conditions
  \item System tests are performed every week in standard conditions, in order to guarantee performances and by regression to solve eventual bugs due to software development
\end{itemize}

\subsubsection{The system is controlled and monitored}
\begin{itemize}
  \item The system is equipped with condition-monitoring algorithms 
  \item The condition-monitoring algorithms identify the functions that cause alarm
  \item A unit test is made as soon as a function is identified as potentially at risk
  \item An integration test is performed after the unit test of the potentially at risk function
  \item A system test is performed after the integration and unit test of the potentially at risk function
\end{itemize}

\subsubsection{The code must be clean and easy to understand}
\begin{itemize}
  \item Code must have a clear structure
  \item It is advisable to following the design patterns to provide a standard terminology and to make the software more adaptable to future extensions and improvements
  \item Complete and detailed documentation from a unit to the superior levels is mandatory in order to keep the maintainability on the highest level on the whole system
\end{itemize}

\subsection{Portability}
\begin{itemize}
  \item The software shall run on Windows, IOS, and Linux
  \item The software must be developed in Java
  \item The software can be developed using the standard API’s that span different types of operating systems
  \item The software will be transferred with no modifications to support the environment on another destination machine
  \item Binary translation must be supported by the system as adaptability is significant and practical for a program in “binary” (executable) form
  \item The architecture must be flexible
\end{itemize}


  % 3 - Specific requirements
  \chapter{Specific requirements}
  \section{External Interface Requirements}

\subsection{User Interface}
The following mock ups represent a basic idea of how the Mobile Application
and the Web Interface are supposed to look like.
Through the smart-phone application and web interface, users have access to complete SafeStreets functionalities, including the possibility of notifying violations.

\begin{figure}[H]
  \centering
  \includegraphics[width=\textwidth,height=.95\textheight,keepaspectratio]{RASD_Images/UserInterface/All.jpg}
  \caption{\textit{User interface mock ups}}
\end{figure}

\subsection{Hardware Interfaces}
Since the application runs over the Internet, the hardware of both the server and the client needs to be able to connect to the Internet.
\begin{itemize}
  \item Server-side: the hosting platform will provide all the necessary hardware interface.
  \item Client-side: e.g. Wi-Fi, 3G/4G. 
\end{itemize}
Client-side hardware should have a camera in order to have access to all functionalities of the application. It is also recommended, although not mandatory, that client-side hardware is provided with a GPS.


\subsection{Software Interface}
An API provided by SafeStreets allows third parties to access the read-only functionalities offered by the system. This way, third party companies and applications can have access and possibly embed in their software statistical data about accidents and street safety collected by SafeStreets. 
On the other hand, functionalities that involve the creation of new data are meant to be used only within SafeStreets official platforms and are not exposed to third parties.

\subsection{Communication Interface}
SafeStreets will communicate with third parties and with the client application using the standard HTTPS protocol, which guarantees encryption by default.
\section{Functional Requirements}

\subsection{R1: Allow a visitor to register on the mobile phone app or the web app}
\begin{itemize}
    \item The system must allow the visitor to provide credentials and personal data
    \item The system must verify the correspondence between the ID number provided by the visitor and their personal information
    \item The system must allow the visitor to verify the account with an e-mail or SMS verification code
    \item The system must send a recap of registration to email of the visitor
    \item The system must verify that there are no other registered users or authorities with the same e-mail or ID number
    \item The user to accept users data privacy conditions to successfully register to the system
\end{itemize}
\subsubsection{Scenario 1}
Mario Rossi saw the advertisement of SafeStreets and decided to download the mobile application in order to be allowed to notify violations and improve safety of streets in his city. After
opening the new app, he is asked to fill a form with all his personal information, full name, email address, mobile number, ID card, and phone number. To proceed with his
registration, after all having filled all the form, Mario clicks on the "Create account" button. The system verify the correspondence between inserted personal data and the ID card information,
and if the verification is positive, Mario is asked whether he prefers to confirm his registration by phone number or by email.
He chooses the alternative he prefers, then he immediately receives the verification code on the account he chose.
He can confirm his registration by clicking on the verification code. The system then sends a recap to notify the positive outcome of registration. Mario is now a Registered
User of SafeStreets; he can login to notify violations on the streets and is enabled to the use of all the functions for a basic user (not an authority) that the system provides.

\subsection{Allow a user to send pictures about street and parking violations}
\begin{itemize}
    \item A user can submit a picture to the system whenever they see a street violation
    \item SafeStreets analyses the image to read the license plait of the vehicle
    \item The user can manually attach the license plait to the image, to help the system with its validation
    \item The user must provide the type of violation, either selecting it from a list or manually typing it if it is not already in the system
    \item The user can help localizing the violation by inserting the name of the street where it occurred
\end{itemize}
\subsubsection{Scenario 2}
Mario Rossi is a user of SafeStreets, he has just finished working, and he wants to go back home, but he is prevented by a car, that is in double-row parking with his own. He waits for ten 
minutes to see if the owner of the car is coming, but then decide to notify the violation. He makes two pictures representing the violation, then logs into the mobile application, 
inserting his personal data, then he clicks on the button to notify a violation, he insert time (17.55), position ( Via Botticelli, Viale Romagna, 2, 20133 Milano MI), a brief description 
of the violation, and the two pictures. He is asked to click the category of the notified violation (Double-Row Parking), or to add it, in the field "other" if the type of violation is not still 
present in the system. He is then asked to insert the license plate of the vehicle involved in the violation in the mobile application, he does it, then the app verifies that it 
corresponds to the license plate identified with the photograph.

\subsection{Validation of reported violations}
\begin{itemize}
    \item Whenever a violation is reported by a user, it is sent to a random group of $k$ SafeStreets users
    \item Users who receive the notification about this approval process are then asked to approve or reject the report or declare neutral if they are not able to verify
    \item The report is validated if and only if the algebraic sum of approvals (+1), neutrality (0) and rejections (-1) is at least $v$
    \item Every user is supposed to respond, the system expects to receive k answers
    \item If the answers of the users for validation, after a settled timeout, are not all received, the answer is automatically considered neutral
\end{itemize}
\subsubsection{Scenario 3}
SafeStreets has received the notification of Mario Rossi regarding a violation of double-row parking. To verify the violation, it sends a request to a 
number k of users. The number of users is randomly chosen, and depend from the number of users logged in the application at the time the system received the report of Mario Rossi
and the users that were logged just a few time before, to increase probability they will be available to answer the report. K users receive the request for validation, and they start 
answering the report. The system receives reports and makes the algebraic sum of approvals and rejections. The calculated sum is equal to $m$ and comparing the values, the result of the 
inequation $m$>$v$ is positive; as a consequence, the system validates the report and registers the violation.

\subsection{Allow a user to see the areas where a violation is more likely to happen}
\begin{itemize}
    \item SafeStreets can aggregate data inputted by users to show relevant statistics about the frequency of street violations
    \item Users can see how many violations usually happen in their neighborhood, in their current location or in an area of their choice
    \item A user has access to the type and amount of violations that happened
\end{itemize}
\subsubsection{Scenario 4}
Mario Rossi searches for a language school to improve English for his daughter, he found different schools in his city on the internet, but he wants an area 
not too far from his house where the little girl can securely go also on foot. He has a list of five schools to decide. He is a user of SafeStreets, so 
for every school in his list, he looks on the web interface of the application for the statistics about car accidents in all the five zones. He insert the 
address of the school, and the system responds with a map, showing the near unsafe area if the school is in a safe area, or, if the school is in an unsafe 
area, showing more detailed data about frequency of car accidents and violations. This operation is repeated five times.

\subsection{Register the violation}
\begin{itemize}
    \item The reported violation must have been validated by the system
    \item The system checks the type of violation
    \item In case the violation is already supported by the system, the number of violations of that type is increased, the license plate of the involved vehicle is added to the list of offenders, and in a report that SafeStreets will deliver to authorities
    \item In case the violation is not supported by the system, the system updates the list of supported violations, then proceed as in the previous case
\end{itemize}
\subsubsection{Scenario 5}
Luca Bianchi is in his lunch break, with his colleagues, and he is waiting for crossing the road on his way to go back to the office. While stopped at the semaphore

\subsection{Allow authorities to see data about violations and who committed them}
\begin{itemize}
    \item Authorities can access all data about violations that a standard user can access
    \item Authorities also have access to more specific data about who committed a violation, like the license plait of the car or how many infractions have been associated to a specific car
\end{itemize}

\subsection{Allow any citizen to become a user by providing a document}
\begin{itemize}
    \item A citizen can register to SafeStreets by providing their ID number
    \item The system must allow a user to register with an email and a password, that will be asked every time they log in
\end{itemize}

\subsection{Allow authorities to register with a special user profile}
\begin{itemize}
    \item Authorities must first register as citizens
    \item A standard user profile can be upgraded to authority profile if it is verified by the system
    \item To obtain privileged access, the user must provide a valid document that proves their authority status
\end{itemize}

\subsection{Unsafe Areas Identification}
\begin{itemize}
  \item SafeStreets can cross information from different sources to identify potentially unsafe areas
  \item SafeStreets may be integrated with a public service, offered by the municipality, that provides such information
  \item Once identified, SafeStreets can suggest possible interventions to prevent accidents
\end{itemize}

\subsection{Traffic Ticket Generation}
\begin{itemize}
  \item SafeStreets can generate tickets for traffic violations reported from registered users
  \item The information is kept safe and intact through the chain of custody, from the end user to the local police officer issuing the tickets
\end{itemize}
\section{Performance Requirements}

\subsection{Reliability}
\subsubsection{Index MTBF}
\begin{itemize}
    \item 90\% of all response time should be less than 4.5 second
    \item The response times will be measured using HP LoadRunner (or similar tool) located behind the firewall and in front of the web servers
    \item Backend response times will be measured using the application server log files
    \item The system is provided to serve a huge number of users in parallel
    \item At the beginning the system is supposed to be correctly and quickly responding to 100'000 users
    \item As number of users is expected to be grown linearly, the system is provided of a flexible structure, to adapt to further number of requests
    \item Daily tests to check performance will be executed to guarantee the service
    \item The performance tests should not exceed 50\% CPU utilization during the execution time
\end{itemize}
\section{Design Constraints}

\subsection{Standards compliance}
  \begin{itemize}
    \item To guarantee the compatibility with the potential municipality accident information service, our backend software should communicate and be able to read data in a portable format like JSON or XML
    \item To transfer this data, the software should expose an HTTP REST API
  \end{itemize}

\subsection{Hardware limitations}
  \begin{itemize}
    \item The frontend application should be as lightweight as possible to support the diversity of mobile device hardware
    \item Devices with a camera resolution of less than 2MP should be marked as incompatible with SafeStreets due to the poor quality of the pictures that are taken with them
  \end{itemize}

\subsection{Other constraints}
  \begin{itemize}
    \item The language used to develop the backend application will be chosen from the ones that are most supported by the main cloud infrastructure providers, in order to have a wider choice of hosting plans
  \end{itemize}
\section{Software System Attributes}

\subsection{Reliability}
\subsubsection{Index MTBF}
\begin{itemize}
    \item Index MTBF must consider as failures those out of design conditions which place the system out of service, for example by an overload of user’s requests
    \item MTBF is of 600'000 hours
\end{itemize}

\subsubsection{Index MTTR}
\begin{itemize}
    \item Index MTTR must consider time of testing and eventual solution of bugs
    \item MTTR must be of two hours
\end{itemize}

\subsubsection{The system must verify position in an accurate way}
\begin{itemize}
    \item The system expects to receive two pictures			
    \item The received pictures must be of at least 2 Mp
    \item The first picture must represent a general overview of the violation, to identify street and area
    \item The second picture must allow the system to identify correctly the license plate, if the violation includes a car
    \item The second picture is simply an additive detail, if the violation doesn't include a car
    \item In the second picture, if the violation includes a car, the license plate must occupy at least a quarter of the total picture
    \item The system must verify the position provided by the user that sent a picture notifying a violation by comparison to the data stored and the GPS service
    \item The identification of coordinates must not be less precise than 3 metres in case of parking violation
\end{itemize}

\subsubsection{Availability}
\begin{itemize}
    \item The system must guarantee a 24/7 service
    \item The system must be available 99.99\%
    \item The system must be available when performing the standard routine tests for maintenance
    \item The system can bear slowdowns of the service in case of extraordinary maintenance due to overloading of requests
\end{itemize}

\subsection{Security}
\subsubsection{Secrecy of data received and of users information}
\begin{itemize}
    \item The system should never allow both non-registered users and registered users to see the identity of the person that notifies the violation
    \item User credentials, ID and password, have to be encrypted and safely stored
    \item Every user has to register with username and password, but the system provides for each user a multi-factor authentication
\end{itemize}

\subsubsection{Integrity of data}
\begin{itemize}
    \item The information sent by users (picture, date, time and position of the violation) has to be encrypted in order to keep the integrity of message, and to avoid manipulation of data
\end{itemize}

\subsubsection{Measures of security according to danger level}
\begin{itemize}
  \item Public statistics to identify the most dangerous areas are provided for SafeStreets
  \item SafeStreets makes a ranking of dangerousness (minimum, medium, high)
  \item The authorities can decide to improve security by highlighting streets in areas at minimum risk
  \item The authorities can decide to improve security by adding cameras in the areas at medium risk 
  \item The authorities can decide to improve security by both adding cameras and doubling patrol shifts  in the areas at large risk
\end{itemize}

\subsection{Maintainability}
\subsubsection{Testing overview}
\begin{itemize}
  \item The system is checked in its correct functioning by software testing
  \item Unit tests are performed every month in standard conditions
  \item Integration tests are performed every two weeks in standard conditions
  \item System tests are performed every week in standard conditions, in order to guarantee performances and by regression to solve eventual bugs due to software development
\end{itemize}

\subsubsection{The system is controlled and monitored}
\begin{itemize}
  \item The system is equipped with condition-monitoring algorithms 
  \item The condition-monitoring algorithms identify the functions that cause alarm
  \item A unit test is made as soon as a function is identified as potentially at risk
  \item An integration test is performed after the unit test of the potentially at risk function
  \item A system test is performed after the integration and unit test of the potentially at risk function
\end{itemize}

\subsubsection{The code must be clean and easy to understand}
\begin{itemize}
  \item Code must have a clear structure
  \item It is advisable to following the design patterns to provide a standard terminology and to make the software more adaptable to future extensions and improvements
  \item Complete and detailed documentation from a unit to the superior levels is mandatory in order to keep the maintainability on the highest level on the whole system
\end{itemize}

\subsection{Portability}
\begin{itemize}
  \item The software shall run on Windows, IOS, and Linux
  \item The software must be developed in Java
  \item The software can be developed using the standard API’s that span different types of operating systems
  \item The software will be transferred with no modifications to support the environment on another destination machine
  \item Binary translation must be supported by the system as adaptability is significant and practical for a program in “binary” (executable) form
  \item The architecture must be flexible
\end{itemize}

  
  % 4 - Formal using Alloy
  \chapter{Formal using Alloy}
  \section{External Interface Requirements}

\subsection{User Interface}
The following mock ups represent a basic idea of how the Mobile Application
and the Web Interface are supposed to look like.
Through the smart-phone application and web interface, users have access to complete SafeStreets functionalities, including the possibility of notifying violations.

\begin{figure}[H]
  \centering
  \includegraphics[width=\textwidth,height=.95\textheight,keepaspectratio]{RASD_Images/UserInterface/All.jpg}
  \caption{\textit{User interface mock ups}}
\end{figure}

\subsection{Hardware Interfaces}
Since the application runs over the Internet, the hardware of both the server and the client needs to be able to connect to the Internet.
\begin{itemize}
  \item Server-side: the hosting platform will provide all the necessary hardware interface.
  \item Client-side: e.g. Wi-Fi, 3G/4G. 
\end{itemize}
Client-side hardware should have a camera in order to have access to all functionalities of the application. It is also recommended, although not mandatory, that client-side hardware is provided with a GPS.


\subsection{Software Interface}
An API provided by SafeStreets allows third parties to access the read-only functionalities offered by the system. This way, third party companies and applications can have access and possibly embed in their software statistical data about accidents and street safety collected by SafeStreets. 
On the other hand, functionalities that involve the creation of new data are meant to be used only within SafeStreets official platforms and are not exposed to third parties.

\subsection{Communication Interface}
SafeStreets will communicate with third parties and with the client application using the standard HTTPS protocol, which guarantees encryption by default.
\section{Functional Requirements}

\subsection{R1: Allow a visitor to register on the mobile phone app or the web app}
\begin{itemize}
    \item The system must allow the visitor to provide credentials and personal data
    \item The system must verify the correspondence between the ID number provided by the visitor and their personal information
    \item The system must allow the visitor to verify the account with an e-mail or SMS verification code
    \item The system must send a recap of registration to email of the visitor
    \item The system must verify that there are no other registered users or authorities with the same e-mail or ID number
    \item The user to accept users data privacy conditions to successfully register to the system
\end{itemize}
\subsubsection{Scenario 1}
Mario Rossi saw the advertisement of SafeStreets and decided to download the mobile application in order to be allowed to notify violations and improve safety of streets in his city. After
opening the new app, he is asked to fill a form with all his personal information, full name, email address, mobile number, ID card, and phone number. To proceed with his
registration, after all having filled all the form, Mario clicks on the "Create account" button. The system verify the correspondence between inserted personal data and the ID card information,
and if the verification is positive, Mario is asked whether he prefers to confirm his registration by phone number or by email.
He chooses the alternative he prefers, then he immediately receives the verification code on the account he chose.
He can confirm his registration by clicking on the verification code. The system then sends a recap to notify the positive outcome of registration. Mario is now a Registered
User of SafeStreets; he can login to notify violations on the streets and is enabled to the use of all the functions for a basic user (not an authority) that the system provides.

\subsection{Allow a user to send pictures about street and parking violations}
\begin{itemize}
    \item A user can submit a picture to the system whenever they see a street violation
    \item SafeStreets analyses the image to read the license plait of the vehicle
    \item The user can manually attach the license plait to the image, to help the system with its validation
    \item The user must provide the type of violation, either selecting it from a list or manually typing it if it is not already in the system
    \item The user can help localizing the violation by inserting the name of the street where it occurred
\end{itemize}
\subsubsection{Scenario 2}
Mario Rossi is a user of SafeStreets, he has just finished working, and he wants to go back home, but he is prevented by a car, that is in double-row parking with his own. He waits for ten 
minutes to see if the owner of the car is coming, but then decide to notify the violation. He makes two pictures representing the violation, then logs into the mobile application, 
inserting his personal data, then he clicks on the button to notify a violation, he insert time (17.55), position ( Via Botticelli, Viale Romagna, 2, 20133 Milano MI), a brief description 
of the violation, and the two pictures. He is asked to click the category of the notified violation (Double-Row Parking), or to add it, in the field "other" if the type of violation is not still 
present in the system. He is then asked to insert the license plate of the vehicle involved in the violation in the mobile application, he does it, then the app verifies that it 
corresponds to the license plate identified with the photograph.

\subsection{Validation of reported violations}
\begin{itemize}
    \item Whenever a violation is reported by a user, it is sent to a random group of $k$ SafeStreets users
    \item Users who receive the notification about this approval process are then asked to approve or reject the report or declare neutral if they are not able to verify
    \item The report is validated if and only if the algebraic sum of approvals (+1), neutrality (0) and rejections (-1) is at least $v$
    \item Every user is supposed to respond, the system expects to receive k answers
    \item If the answers of the users for validation, after a settled timeout, are not all received, the answer is automatically considered neutral
\end{itemize}
\subsubsection{Scenario 3}
SafeStreets has received the notification of Mario Rossi regarding a violation of double-row parking. To verify the violation, it sends a request to a 
number k of users. The number of users is randomly chosen, and depend from the number of users logged in the application at the time the system received the report of Mario Rossi
and the users that were logged just a few time before, to increase probability they will be available to answer the report. K users receive the request for validation, and they start 
answering the report. The system receives reports and makes the algebraic sum of approvals and rejections. The calculated sum is equal to $m$ and comparing the values, the result of the 
inequation $m$>$v$ is positive; as a consequence, the system validates the report and registers the violation.

\subsection{Allow a user to see the areas where a violation is more likely to happen}
\begin{itemize}
    \item SafeStreets can aggregate data inputted by users to show relevant statistics about the frequency of street violations
    \item Users can see how many violations usually happen in their neighborhood, in their current location or in an area of their choice
    \item A user has access to the type and amount of violations that happened
\end{itemize}
\subsubsection{Scenario 4}
Mario Rossi searches for a language school to improve English for his daughter, he found different schools in his city on the internet, but he wants an area 
not too far from his house where the little girl can securely go also on foot. He has a list of five schools to decide. He is a user of SafeStreets, so 
for every school in his list, he looks on the web interface of the application for the statistics about car accidents in all the five zones. He insert the 
address of the school, and the system responds with a map, showing the near unsafe area if the school is in a safe area, or, if the school is in an unsafe 
area, showing more detailed data about frequency of car accidents and violations. This operation is repeated five times.

\subsection{Register the violation}
\begin{itemize}
    \item The reported violation must have been validated by the system
    \item The system checks the type of violation
    \item In case the violation is already supported by the system, the number of violations of that type is increased, the license plate of the involved vehicle is added to the list of offenders, and in a report that SafeStreets will deliver to authorities
    \item In case the violation is not supported by the system, the system updates the list of supported violations, then proceed as in the previous case
\end{itemize}
\subsubsection{Scenario 5}
Luca Bianchi is in his lunch break, with his colleagues, and he is waiting for crossing the road on his way to go back to the office. While stopped at the semaphore

\subsection{Allow authorities to see data about violations and who committed them}
\begin{itemize}
    \item Authorities can access all data about violations that a standard user can access
    \item Authorities also have access to more specific data about who committed a violation, like the license plait of the car or how many infractions have been associated to a specific car
\end{itemize}

\subsection{Allow any citizen to become a user by providing a document}
\begin{itemize}
    \item A citizen can register to SafeStreets by providing their ID number
    \item The system must allow a user to register with an email and a password, that will be asked every time they log in
\end{itemize}

\subsection{Allow authorities to register with a special user profile}
\begin{itemize}
    \item Authorities must first register as citizens
    \item A standard user profile can be upgraded to authority profile if it is verified by the system
    \item To obtain privileged access, the user must provide a valid document that proves their authority status
\end{itemize}

\subsection{Unsafe Areas Identification}
\begin{itemize}
  \item SafeStreets can cross information from different sources to identify potentially unsafe areas
  \item SafeStreets may be integrated with a public service, offered by the municipality, that provides such information
  \item Once identified, SafeStreets can suggest possible interventions to prevent accidents
\end{itemize}

\subsection{Traffic Ticket Generation}
\begin{itemize}
  \item SafeStreets can generate tickets for traffic violations reported from registered users
  \item The information is kept safe and intact through the chain of custody, from the end user to the local police officer issuing the tickets
\end{itemize}
\section{Performance Requirements}

\subsection{Reliability}
\subsubsection{Index MTBF}
\begin{itemize}
    \item 90\% of all response time should be less than 4.5 second
    \item The response times will be measured using HP LoadRunner (or similar tool) located behind the firewall and in front of the web servers
    \item Backend response times will be measured using the application server log files
    \item The system is provided to serve a huge number of users in parallel
    \item At the beginning the system is supposed to be correctly and quickly responding to 100'000 users
    \item As number of users is expected to be grown linearly, the system is provided of a flexible structure, to adapt to further number of requests
    \item Daily tests to check performance will be executed to guarantee the service
    \item The performance tests should not exceed 50\% CPU utilization during the execution time
\end{itemize}
\section{Design Constraints}

\subsection{Standards compliance}
  \begin{itemize}
    \item To guarantee the compatibility with the potential municipality accident information service, our backend software should communicate and be able to read data in a portable format like JSON or XML
    \item To transfer this data, the software should expose an HTTP REST API
  \end{itemize}

\subsection{Hardware limitations}
  \begin{itemize}
    \item The frontend application should be as lightweight as possible to support the diversity of mobile device hardware
    \item Devices with a camera resolution of less than 2MP should be marked as incompatible with SafeStreets due to the poor quality of the pictures that are taken with them
  \end{itemize}

\subsection{Other constraints}
  \begin{itemize}
    \item The language used to develop the backend application will be chosen from the ones that are most supported by the main cloud infrastructure providers, in order to have a wider choice of hosting plans
  \end{itemize}
\section{Software System Attributes}

\subsection{Reliability}
\subsubsection{Index MTBF}
\begin{itemize}
    \item Index MTBF must consider as failures those out of design conditions which place the system out of service, for example by an overload of user’s requests
    \item MTBF is of 600'000 hours
\end{itemize}

\subsubsection{Index MTTR}
\begin{itemize}
    \item Index MTTR must consider time of testing and eventual solution of bugs
    \item MTTR must be of two hours
\end{itemize}

\subsubsection{The system must verify position in an accurate way}
\begin{itemize}
    \item The system expects to receive two pictures			
    \item The received pictures must be of at least 2 Mp
    \item The first picture must represent a general overview of the violation, to identify street and area
    \item The second picture must allow the system to identify correctly the license plate, if the violation includes a car
    \item The second picture is simply an additive detail, if the violation doesn't include a car
    \item In the second picture, if the violation includes a car, the license plate must occupy at least a quarter of the total picture
    \item The system must verify the position provided by the user that sent a picture notifying a violation by comparison to the data stored and the GPS service
    \item The identification of coordinates must not be less precise than 3 metres in case of parking violation
\end{itemize}

\subsubsection{Availability}
\begin{itemize}
    \item The system must guarantee a 24/7 service
    \item The system must be available 99.99\%
    \item The system must be available when performing the standard routine tests for maintenance
    \item The system can bear slowdowns of the service in case of extraordinary maintenance due to overloading of requests
\end{itemize}

\subsection{Security}
\subsubsection{Secrecy of data received and of users information}
\begin{itemize}
    \item The system should never allow both non-registered users and registered users to see the identity of the person that notifies the violation
    \item User credentials, ID and password, have to be encrypted and safely stored
    \item Every user has to register with username and password, but the system provides for each user a multi-factor authentication
\end{itemize}

\subsubsection{Integrity of data}
\begin{itemize}
    \item The information sent by users (picture, date, time and position of the violation) has to be encrypted in order to keep the integrity of message, and to avoid manipulation of data
\end{itemize}

\subsubsection{Measures of security according to danger level}
\begin{itemize}
  \item Public statistics to identify the most dangerous areas are provided for SafeStreets
  \item SafeStreets makes a ranking of dangerousness (minimum, medium, high)
  \item The authorities can decide to improve security by highlighting streets in areas at minimum risk
  \item The authorities can decide to improve security by adding cameras in the areas at medium risk 
  \item The authorities can decide to improve security by both adding cameras and doubling patrol shifts  in the areas at large risk
\end{itemize}

\subsection{Maintainability}
\subsubsection{Testing overview}
\begin{itemize}
  \item The system is checked in its correct functioning by software testing
  \item Unit tests are performed every month in standard conditions
  \item Integration tests are performed every two weeks in standard conditions
  \item System tests are performed every week in standard conditions, in order to guarantee performances and by regression to solve eventual bugs due to software development
\end{itemize}

\subsubsection{The system is controlled and monitored}
\begin{itemize}
  \item The system is equipped with condition-monitoring algorithms 
  \item The condition-monitoring algorithms identify the functions that cause alarm
  \item A unit test is made as soon as a function is identified as potentially at risk
  \item An integration test is performed after the unit test of the potentially at risk function
  \item A system test is performed after the integration and unit test of the potentially at risk function
\end{itemize}

\subsubsection{The code must be clean and easy to understand}
\begin{itemize}
  \item Code must have a clear structure
  \item It is advisable to following the design patterns to provide a standard terminology and to make the software more adaptable to future extensions and improvements
  \item Complete and detailed documentation from a unit to the superior levels is mandatory in order to keep the maintainability on the highest level on the whole system
\end{itemize}

\subsection{Portability}
\begin{itemize}
  \item The software shall run on Windows, IOS, and Linux
  \item The software must be developed in Java
  \item The software can be developed using the standard API’s that span different types of operating systems
  \item The software will be transferred with no modifications to support the environment on another destination machine
  \item Binary translation must be supported by the system as adaptability is significant and practical for a program in “binary” (executable) form
  \item The architecture must be flexible
\end{itemize}


  % 5 - Effort spent
  \chapter{Effort spent}
  \section{External Interface Requirements}

\subsection{User Interface}
The following mock ups represent a basic idea of how the Mobile Application
and the Web Interface are supposed to look like.
Through the smart-phone application and web interface, users have access to complete SafeStreets functionalities, including the possibility of notifying violations.

\begin{figure}[H]
  \centering
  \includegraphics[width=\textwidth,height=.95\textheight,keepaspectratio]{RASD_Images/UserInterface/All.jpg}
  \caption{\textit{User interface mock ups}}
\end{figure}

\subsection{Hardware Interfaces}
Since the application runs over the Internet, the hardware of both the server and the client needs to be able to connect to the Internet.
\begin{itemize}
  \item Server-side: the hosting platform will provide all the necessary hardware interface.
  \item Client-side: e.g. Wi-Fi, 3G/4G. 
\end{itemize}
Client-side hardware should have a camera in order to have access to all functionalities of the application. It is also recommended, although not mandatory, that client-side hardware is provided with a GPS.


\subsection{Software Interface}
An API provided by SafeStreets allows third parties to access the read-only functionalities offered by the system. This way, third party companies and applications can have access and possibly embed in their software statistical data about accidents and street safety collected by SafeStreets. 
On the other hand, functionalities that involve the creation of new data are meant to be used only within SafeStreets official platforms and are not exposed to third parties.

\subsection{Communication Interface}
SafeStreets will communicate with third parties and with the client application using the standard HTTPS protocol, which guarantees encryption by default.
\section{Functional Requirements}

\subsection{R1: Allow a visitor to register on the mobile phone app or the web app}
\begin{itemize}
    \item The system must allow the visitor to provide credentials and personal data
    \item The system must verify the correspondence between the ID number provided by the visitor and their personal information
    \item The system must allow the visitor to verify the account with an e-mail or SMS verification code
    \item The system must send a recap of registration to email of the visitor
    \item The system must verify that there are no other registered users or authorities with the same e-mail or ID number
    \item The user to accept users data privacy conditions to successfully register to the system
\end{itemize}
\subsubsection{Scenario 1}
Mario Rossi saw the advertisement of SafeStreets and decided to download the mobile application in order to be allowed to notify violations and improve safety of streets in his city. After
opening the new app, he is asked to fill a form with all his personal information, full name, email address, mobile number, ID card, and phone number. To proceed with his
registration, after all having filled all the form, Mario clicks on the "Create account" button. The system verify the correspondence between inserted personal data and the ID card information,
and if the verification is positive, Mario is asked whether he prefers to confirm his registration by phone number or by email.
He chooses the alternative he prefers, then he immediately receives the verification code on the account he chose.
He can confirm his registration by clicking on the verification code. The system then sends a recap to notify the positive outcome of registration. Mario is now a Registered
User of SafeStreets; he can login to notify violations on the streets and is enabled to the use of all the functions for a basic user (not an authority) that the system provides.

\subsection{Allow a user to send pictures about street and parking violations}
\begin{itemize}
    \item A user can submit a picture to the system whenever they see a street violation
    \item SafeStreets analyses the image to read the license plait of the vehicle
    \item The user can manually attach the license plait to the image, to help the system with its validation
    \item The user must provide the type of violation, either selecting it from a list or manually typing it if it is not already in the system
    \item The user can help localizing the violation by inserting the name of the street where it occurred
\end{itemize}
\subsubsection{Scenario 2}
Mario Rossi is a user of SafeStreets, he has just finished working, and he wants to go back home, but he is prevented by a car, that is in double-row parking with his own. He waits for ten 
minutes to see if the owner of the car is coming, but then decide to notify the violation. He makes two pictures representing the violation, then logs into the mobile application, 
inserting his personal data, then he clicks on the button to notify a violation, he insert time (17.55), position ( Via Botticelli, Viale Romagna, 2, 20133 Milano MI), a brief description 
of the violation, and the two pictures. He is asked to click the category of the notified violation (Double-Row Parking), or to add it, in the field "other" if the type of violation is not still 
present in the system. He is then asked to insert the license plate of the vehicle involved in the violation in the mobile application, he does it, then the app verifies that it 
corresponds to the license plate identified with the photograph.

\subsection{Validation of reported violations}
\begin{itemize}
    \item Whenever a violation is reported by a user, it is sent to a random group of $k$ SafeStreets users
    \item Users who receive the notification about this approval process are then asked to approve or reject the report or declare neutral if they are not able to verify
    \item The report is validated if and only if the algebraic sum of approvals (+1), neutrality (0) and rejections (-1) is at least $v$
    \item Every user is supposed to respond, the system expects to receive k answers
    \item If the answers of the users for validation, after a settled timeout, are not all received, the answer is automatically considered neutral
\end{itemize}
\subsubsection{Scenario 3}
SafeStreets has received the notification of Mario Rossi regarding a violation of double-row parking. To verify the violation, it sends a request to a 
number k of users. The number of users is randomly chosen, and depend from the number of users logged in the application at the time the system received the report of Mario Rossi
and the users that were logged just a few time before, to increase probability they will be available to answer the report. K users receive the request for validation, and they start 
answering the report. The system receives reports and makes the algebraic sum of approvals and rejections. The calculated sum is equal to $m$ and comparing the values, the result of the 
inequation $m$>$v$ is positive; as a consequence, the system validates the report and registers the violation.

\subsection{Allow a user to see the areas where a violation is more likely to happen}
\begin{itemize}
    \item SafeStreets can aggregate data inputted by users to show relevant statistics about the frequency of street violations
    \item Users can see how many violations usually happen in their neighborhood, in their current location or in an area of their choice
    \item A user has access to the type and amount of violations that happened
\end{itemize}
\subsubsection{Scenario 4}
Mario Rossi searches for a language school to improve English for his daughter, he found different schools in his city on the internet, but he wants an area 
not too far from his house where the little girl can securely go also on foot. He has a list of five schools to decide. He is a user of SafeStreets, so 
for every school in his list, he looks on the web interface of the application for the statistics about car accidents in all the five zones. He insert the 
address of the school, and the system responds with a map, showing the near unsafe area if the school is in a safe area, or, if the school is in an unsafe 
area, showing more detailed data about frequency of car accidents and violations. This operation is repeated five times.

\subsection{Register the violation}
\begin{itemize}
    \item The reported violation must have been validated by the system
    \item The system checks the type of violation
    \item In case the violation is already supported by the system, the number of violations of that type is increased, the license plate of the involved vehicle is added to the list of offenders, and in a report that SafeStreets will deliver to authorities
    \item In case the violation is not supported by the system, the system updates the list of supported violations, then proceed as in the previous case
\end{itemize}
\subsubsection{Scenario 5}
Luca Bianchi is in his lunch break, with his colleagues, and he is waiting for crossing the road on his way to go back to the office. While stopped at the semaphore

\subsection{Allow authorities to see data about violations and who committed them}
\begin{itemize}
    \item Authorities can access all data about violations that a standard user can access
    \item Authorities also have access to more specific data about who committed a violation, like the license plait of the car or how many infractions have been associated to a specific car
\end{itemize}

\subsection{Allow any citizen to become a user by providing a document}
\begin{itemize}
    \item A citizen can register to SafeStreets by providing their ID number
    \item The system must allow a user to register with an email and a password, that will be asked every time they log in
\end{itemize}

\subsection{Allow authorities to register with a special user profile}
\begin{itemize}
    \item Authorities must first register as citizens
    \item A standard user profile can be upgraded to authority profile if it is verified by the system
    \item To obtain privileged access, the user must provide a valid document that proves their authority status
\end{itemize}

\subsection{Unsafe Areas Identification}
\begin{itemize}
  \item SafeStreets can cross information from different sources to identify potentially unsafe areas
  \item SafeStreets may be integrated with a public service, offered by the municipality, that provides such information
  \item Once identified, SafeStreets can suggest possible interventions to prevent accidents
\end{itemize}

\subsection{Traffic Ticket Generation}
\begin{itemize}
  \item SafeStreets can generate tickets for traffic violations reported from registered users
  \item The information is kept safe and intact through the chain of custody, from the end user to the local police officer issuing the tickets
\end{itemize}
\section{Performance Requirements}

\subsection{Reliability}
\subsubsection{Index MTBF}
\begin{itemize}
    \item 90\% of all response time should be less than 4.5 second
    \item The response times will be measured using HP LoadRunner (or similar tool) located behind the firewall and in front of the web servers
    \item Backend response times will be measured using the application server log files
    \item The system is provided to serve a huge number of users in parallel
    \item At the beginning the system is supposed to be correctly and quickly responding to 100'000 users
    \item As number of users is expected to be grown linearly, the system is provided of a flexible structure, to adapt to further number of requests
    \item Daily tests to check performance will be executed to guarantee the service
    \item The performance tests should not exceed 50\% CPU utilization during the execution time
\end{itemize}
\section{Design Constraints}

\subsection{Standards compliance}
  \begin{itemize}
    \item To guarantee the compatibility with the potential municipality accident information service, our backend software should communicate and be able to read data in a portable format like JSON or XML
    \item To transfer this data, the software should expose an HTTP REST API
  \end{itemize}

\subsection{Hardware limitations}
  \begin{itemize}
    \item The frontend application should be as lightweight as possible to support the diversity of mobile device hardware
    \item Devices with a camera resolution of less than 2MP should be marked as incompatible with SafeStreets due to the poor quality of the pictures that are taken with them
  \end{itemize}

\subsection{Other constraints}
  \begin{itemize}
    \item The language used to develop the backend application will be chosen from the ones that are most supported by the main cloud infrastructure providers, in order to have a wider choice of hosting plans
  \end{itemize}
\section{Software System Attributes}

\subsection{Reliability}
\subsubsection{Index MTBF}
\begin{itemize}
    \item Index MTBF must consider as failures those out of design conditions which place the system out of service, for example by an overload of user’s requests
    \item MTBF is of 600'000 hours
\end{itemize}

\subsubsection{Index MTTR}
\begin{itemize}
    \item Index MTTR must consider time of testing and eventual solution of bugs
    \item MTTR must be of two hours
\end{itemize}

\subsubsection{The system must verify position in an accurate way}
\begin{itemize}
    \item The system expects to receive two pictures			
    \item The received pictures must be of at least 2 Mp
    \item The first picture must represent a general overview of the violation, to identify street and area
    \item The second picture must allow the system to identify correctly the license plate, if the violation includes a car
    \item The second picture is simply an additive detail, if the violation doesn't include a car
    \item In the second picture, if the violation includes a car, the license plate must occupy at least a quarter of the total picture
    \item The system must verify the position provided by the user that sent a picture notifying a violation by comparison to the data stored and the GPS service
    \item The identification of coordinates must not be less precise than 3 metres in case of parking violation
\end{itemize}

\subsubsection{Availability}
\begin{itemize}
    \item The system must guarantee a 24/7 service
    \item The system must be available 99.99\%
    \item The system must be available when performing the standard routine tests for maintenance
    \item The system can bear slowdowns of the service in case of extraordinary maintenance due to overloading of requests
\end{itemize}

\subsection{Security}
\subsubsection{Secrecy of data received and of users information}
\begin{itemize}
    \item The system should never allow both non-registered users and registered users to see the identity of the person that notifies the violation
    \item User credentials, ID and password, have to be encrypted and safely stored
    \item Every user has to register with username and password, but the system provides for each user a multi-factor authentication
\end{itemize}

\subsubsection{Integrity of data}
\begin{itemize}
    \item The information sent by users (picture, date, time and position of the violation) has to be encrypted in order to keep the integrity of message, and to avoid manipulation of data
\end{itemize}

\subsubsection{Measures of security according to danger level}
\begin{itemize}
  \item Public statistics to identify the most dangerous areas are provided for SafeStreets
  \item SafeStreets makes a ranking of dangerousness (minimum, medium, high)
  \item The authorities can decide to improve security by highlighting streets in areas at minimum risk
  \item The authorities can decide to improve security by adding cameras in the areas at medium risk 
  \item The authorities can decide to improve security by both adding cameras and doubling patrol shifts  in the areas at large risk
\end{itemize}

\subsection{Maintainability}
\subsubsection{Testing overview}
\begin{itemize}
  \item The system is checked in its correct functioning by software testing
  \item Unit tests are performed every month in standard conditions
  \item Integration tests are performed every two weeks in standard conditions
  \item System tests are performed every week in standard conditions, in order to guarantee performances and by regression to solve eventual bugs due to software development
\end{itemize}

\subsubsection{The system is controlled and monitored}
\begin{itemize}
  \item The system is equipped with condition-monitoring algorithms 
  \item The condition-monitoring algorithms identify the functions that cause alarm
  \item A unit test is made as soon as a function is identified as potentially at risk
  \item An integration test is performed after the unit test of the potentially at risk function
  \item A system test is performed after the integration and unit test of the potentially at risk function
\end{itemize}

\subsubsection{The code must be clean and easy to understand}
\begin{itemize}
  \item Code must have a clear structure
  \item It is advisable to following the design patterns to provide a standard terminology and to make the software more adaptable to future extensions and improvements
  \item Complete and detailed documentation from a unit to the superior levels is mandatory in order to keep the maintainability on the highest level on the whole system
\end{itemize}

\subsection{Portability}
\begin{itemize}
  \item The software shall run on Windows, IOS, and Linux
  \item The software must be developed in Java
  \item The software can be developed using the standard API’s that span different types of operating systems
  \item The software will be transferred with no modifications to support the environment on another destination machine
  \item Binary translation must be supported by the system as adaptability is significant and practical for a program in “binary” (executable) form
  \item The architecture must be flexible
\end{itemize}


  % 6 - References
  \begin{thebibliography}{0}
    \bibitem{latex}\href{https://www.latex-project.org/}{\LaTeX}  for writing the document {\scriptsize \newline$[$\url{https://www.latex-project.org/}$]$}

    \bibitem{alloy-style} \href{https://github.com/Angtrim/alloy-latex-highlighting}{Alloy Latex Highlighter}, a \LaTeX \ package to highlight alloy syntax that we modified to render improve the visualization of some operators
    {\scriptsize \newline$[$\url{https://github.com/Angtrim/alloy-latex-highlighting}$]$}

    \bibitem{title-pic} \href{http://tug.ctan.org/tex-archive/macros/latex/contrib/titlepic/titlepic.sty}{Title Pic}, a \LaTeX \ package that allows customization of the cover page
    {\scriptsize \newline$[$\url{http://tug.ctan.org/tex-archive/macros/latex/contrib/titlepic/titlepic.sty}$]$}

    \bibitem{draw-io} \href{https://www.draw.io/}{draw.io} to draw UML diagrams and sequence diagrams
    {\scriptsize \newline$[$\url{https://www.draw.io/}$]$}

    \bibitem{alloy-analyzer} \href{https://alloytools.org/}{Alloy Analyzer} to generate Alloy worlds and run alloy predicates
    {\scriptsize \newline$[$\url{https://alloytools.org/}$]$}

    \bibitem{tocbibind} \href{https://ctan.org/pkg/tocbibind}{tocbibind}, a \LaTeX \ package to bind the bibliography to the table of contents
    {\scriptsize \newline$[$\url{https://ctan.org/pkg/tocbibind}$]$}

    \bibitem{github} \href{https://github.com}{GitHub} for version control
    {\scriptsize \newline$[$\url{https://github.com}$]$}

    \bibitem{vscode} \href{https://code.visualstudio.com/}{Visual Studio Code} the IDE we used
    {\scriptsize \newline$[$\url{https://code.visualstudio.com/}$]$}

    \bibitem{wikibooks} \href{https://en.wikibooks.org/wiki/LaTeX/}{Wikibooks} we read the section about \LaTeX \ to learn how to structure the document
    {\scriptsize \newline$[$\url{https://en.wikibooks.org/wiki/LaTeX/}$]$}

    \bibitem{stackoverflow} \href{https://tex.stackexchange.com/}{Stack Exchange} for further doubts about \LaTeX
    {\scriptsize \newline$[$\url{https://tex.stackexchange.com/}$]$}

    \bibitem{Ionic} \href{https://ionicframework.com/}{Ionic} to generate mockups
    {\scriptsize \newline$[$\url{https://ionicframework.com/}$]$}

    \bibitem{GIMP} \href{https://www.gimp.org}{GIMP} to edit mockups
    {\scriptsize \newline$[$\url{https://www.gimp.org}$]$}
  \end{thebibliography}
\end{document}