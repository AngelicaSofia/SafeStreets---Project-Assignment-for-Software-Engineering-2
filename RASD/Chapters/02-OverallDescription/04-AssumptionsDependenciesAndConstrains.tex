\section{Assumptions, dependencies and constraints}

\subsection{Domain assumptions}
  \begin{itemize}
    \item $[$D1$]$ Personal information that a visitor has to provide to become a registered user are name, surname, email, ID number and password
    \item $[$D2$]$ Emails are unique and a user can only be associated to one email address
    \item $[$D3$]$ Two-factors authentication can be enabled by the user by providing a phone number (SMS authentication) or by scanning a QR Code with an authenticator app (Token authentication)
    \item $[$D4$]$ Users are enabled to submit reports only after their ID number has been verified by an API offered by public authorities
    \item $[$D5$]$ Citizens must be at least 18 years old to register to SafeStreets
    \item $[$D6$]$ Real world authorities can be uniquely identified through a badge number
    \item $[$D7$]$ Municipalities expose APIs that allow SafeStreets to identify an authority by providing the badge number
    \item $[$D8$]$ Municipalities can expose APIs that allow SafeStreets to retrieve data about other accidents that weren't notified through SafeStreets
    \item $[$D9$]$ Once a SafeStreets report is validated, it has the same legal validity as a report filed personally by an authority, so authorities can emit tickets based on approved reports
    \item $[$D10$]$ The safety level of an area is computed taking into account only street violations and accidents and is based on previous data
    \item $[$D11$]$ No one can alter a report once it has been sent, it can only be validated or rejected
  \end{itemize}

\subsection{Dependencies}
  \begin{itemize}
    \item The authorities will be in charge of every effective measure of security taken to make streets safer
  \end{itemize}
