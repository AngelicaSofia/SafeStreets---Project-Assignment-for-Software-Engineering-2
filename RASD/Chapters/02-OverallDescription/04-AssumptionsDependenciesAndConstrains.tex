\section{Assumptions, dependencies and constraints}

\subsection{Domain assumptions}
  \begin{itemize}
    \item $[$D1$]$ Personal information that a visitor has to provide to become a registered user are name, surname, email, ID number and password
    \item $[$D2$]$ Emails are unique and a user can only be associated to one email address
    \item $[$D3$]$ Two-factors authentication can be enabled by the user by providing a phone number (SMS authentication) or by scanning a QR Code with an authenticator app (Token authentication)
    \item $[$D4$]$ Users are enabled to submit reports only after their ID number has been verified by an API offered by public authorities
    \item $[$D5$]$ Citizens must be at least 18 years old to register to SafeStreets
    \item $[$D6$]$ Real world authorities can be uniquely identified through a badge number
    \item $[$D7$]$ Municipalities expose APIs that allow SafeStreets to identify an authority by providing the badge number
    \item $[$D8$]$ Municipalities can expose APIs that allow SafeStreets to retrieve data about other accidents that weren't notified through SafeStreets
    \item $[$D9$]$ Once a SafeStreets report is validated, it has the same legal validity as a report filed personally by an authority, so authorities can emit tickets based on approved reports
    \item $[$D10$]$ The safety level of an area is computed taking into account only street violations and accidents and is based on previous data
    \item $[$D11$]$ No one can alter a report once it has been sent, it can only be validated or rejected
  \end{itemize}

\subsection{Dependencies}
  \begin{itemize}
    \item SafeStreets provides suggestions for improving the conditions of the most dangerous areas, but the authorities local authorities will be in charge of every effective measure of security taken to make streets safer
    \item Unless the majority of street violations is notified through SafeStreets, the software can only build statistically relevant maps and charts if the municipalities provide APIs that give access to all data about traffic tickets and accidents. This means that the safety of different areas may be based on more or less data and therefore be more or less accurate.
    \item SafeStreets can be used even without GPS, in that case data regarding the location needs to be manually entered by the user: this means that a device with functioning GPS will improve user experience and will also help make sure that the data submitted to the server is consistent.
  \end{itemize}

\subsection{Constraints}
  \begin{itemize}
    \item The service that allows authority to remotely generate traffic tickets works at its best for violations that include a vehicle identifiable by a license plate. If the violation is about a bike or a pedestrian or the license plate for some reason isn't fully recognizable it is necessary that an officers is actually present in the reported location for any further action to be taken.
    \item SafeStreets can work properly and have an impact on the neighborhood only if there is a large amount of engaged users, otherwise only very few reports are submitted and validated. It's therefore necessary for the application to be also entertaining and to provide some rewards to the most dedicated users.
    \item While originally thought for the Italian market, SafeStreets can easily be used and integrated in other countries as long as all citizens have a unique document number and police authorities can also be uniquely identified by a badge number. 
  \end{itemize}
