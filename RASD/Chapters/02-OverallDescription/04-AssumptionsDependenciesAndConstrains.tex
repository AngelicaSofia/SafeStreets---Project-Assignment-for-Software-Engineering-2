\section{Assumptions, dependencies and constraints}

\subsection{Domain assumption}
  \begin{itemize}
    \item Personal information that a visitor has to provide to become a registered user are name, surname, email, ID number and password
    \item Emails are unique and a user can only be associated to one email address
    \item After registration the system sends an email to the user containing a recap of the information they provided and an activation link
    \item Once the activation link is opened in a browser the email address of the user is verified and the user is automatically logged in
    \item Two-factors authentication can be enabled by the user by providing a phone number (SMS authentication) or by scanning a QR Code with an authenticator app (Token authentication)
    \item Once logged in a user can notify traffic violations
    \item Pictures with a resolution of less than 2MP are automatically rejected
    \item --------------------------------------------------- ALT1
    \item An unsafe area, represented as a circle, is described by the coordinate and
    the radius as the maximum acceptable distance from the center of the area
    \item Different unsafe areas do not overlap between them
    \item Every area that is not unsafe is considered safe
    \item Unsafe areas are determined by statistics made by the system, according to the areas where the maximum number of violations occur 
    \item --------------------------------------------------- ALT2
    \item Users can view a custom map that is updated periodically and shows unsafe streets by highlighting them with a gradient of colors from yellow to red which indicates the level of danger
  \end{itemize}

\subsection{Dependencies}
  \begin{itemize}
    \item The authorities will be in charge of every effective measure of security taken to make streets safer
  \end{itemize}
