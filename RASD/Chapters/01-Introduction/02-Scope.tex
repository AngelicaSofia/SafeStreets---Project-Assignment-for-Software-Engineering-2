\section{Scope}
SafeStreets is a service that is available to any citizen having a valid ID number.
The service creates a direct channel to communicate with authorities and with other users about the observed traffic violations, promoting street safety and
helping others who might be in need, for example by reporting a vehicle which is blocking an access for disabled people.
Registered users can submit reports that are going to be validated by other end-users, creating a network of trust.
The reports they file provide information about the violation, the location and the license plate of the subject. This makes reports detailed enough for
approval by public officers, who are able to review them and take proper action to discourage further infractions.

\subsection{World and Shared phenomena}
In this section a list of World and Shared phenomena are shown in order to better distinguish the separation between the system 
(the machine) and the world in which it operates. \newline
\indent \textbf{World phenomena}
\begin{itemize}
  \item A violation occurred
  \item A User decides to report a violation
  \item A User wants to check where violations are more likely to occur
  \item An external organization wants a list of the violations occurred in an area
  \item An Authority acknowledges about the violations
\end{itemize}

\textbf{Shared phenomena}
\begin{itemize}
  \item The user logs in the application
  \item A user reports the violation using the SafeStreets application
  \item A user request the unsafe areas map on the application
  \item Users verify a report by answering the poll generated by a notification of violation
  \item The authority is informed by the system about violations reported in his jurisdiction
  \item The authority generates the traffic ticket for a validated report
\end{itemize}
