\section{External Interface Requirements}

\subsection{User Interface}
The following mockups represent a basic idea of how the Mobile Application
and the Web Interface are supposed to look like.
Users can access to complete SafeStreets functionalities through the smart-phone application, and they will notify violations through it.

On the other side, SafeStreets provides a Web Interface for users. As well as on the mobile phone, 
they can exploit all the functionalities provided by the application, such as the registration or the notification of a violation.

\subsection{Hardware Interfaces}
Since the application must run over the Internet, all the hardware shall re-
quire to connect network will be an hardware interface for the system, both
server and client side.
\begin{itemize}
  \item Server-side: e.g. Modem, WAN - LAN, Ethernet Cross-Cable.
  \item Client-side: e.g. Wi-Fi 802.11ac, 3G/4G. 
\end{itemize}


\subsection{Software Interface}
SafeStreet provide an API, besides the Web Application Interface.
In a more detailed way, it will provide an API that allows third parties to access
the entire set of functionalities provided by the system. In this way 
third party companies and applications can embed in their system the
functionalities concerning with the individual access requests and sampling
requests.

\subsection{Communication Interface}
SafeStreets must rely on the newer TLS version 1.3, released in 2018, to guarantee the best
security possible. This version, in comparison to the previous one (TLS v. 1.2) speeds up encrypted connections;
in fact the handshake phase is executed with only one round-trip, which cuts the encryption latency in half.
The use of TLS version 1.3 is guaranteed at least during the HTTPS connections involving messages
carrying credentials or other sensible data.
