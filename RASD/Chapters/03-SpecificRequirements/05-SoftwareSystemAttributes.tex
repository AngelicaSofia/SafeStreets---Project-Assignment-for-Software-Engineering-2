\section{Software System Attributes}

\subsection{Reliability}
\subsubsection{Index MTBF}
\begin{itemize}
    \item Index MTBF must consider as failures those out of design conditions which place the system out of service, for example by an overload of user's requests
    \item MTBF is of 5,000 hours
\end{itemize}

\subsubsection{Index MTTR}
\begin{itemize}
    \item Index MTTR must consider time of testing and solution of bugs
    \item MTTR must be less than two hours
\end{itemize}

\subsubsection{The position should be verifiable}
\begin{itemize}
    \item The system expects to receive one or two pictures
    \item The received pictures must be of at least 2MP
    \item The first picture must represent a general overview of the violation, to identify the street
    \item The second picture must allow the system to identify correctly the license plate, if the violation includes a vehicle
\end{itemize}

\subsection{Availability}
\begin{itemize}
    \item The system must be available 99.9\%
    \item The system must be available when performing the standard routine tests for maintenance
    \item The system can bear slowdowns of the service in case of extraordinary overload of requests
    \item The system can become unavailable for some seconds due to extraordinary maintenance or major updates
\end{itemize}

\subsection{Security}
\subsubsection{Secrecy of data received and of users' information}
\begin{itemize}
    \item The system should never allow both non-registered users and registered users to see the identity of the person that notifies the violation
    \item User personal data must be encrypted and safely stored
    \item Users' passwords must be salted and hashed before being stored on any persistent medium
    \item Every user can log in using their email and password, but they can optionally enable Two-Factor authentication to improve security
    \item Violation reports must be encrypted before being sent to the server to preserve the secrecy and integrity in the chain of custody
\end{itemize}

\subsubsection{Integrity of data}
\begin{itemize}
    \item The information sent by users (picture, date, time and position of the violation) has to be encrypted in order to keep the integrity and to avoid manipulation of the data
\end{itemize}

\subsubsection{Measures of security according to danger level}
\begin{itemize}
  \item Public statistics to identify the most dangerous areas are provided for SafeStreets
  \item SafeStreets makes a ranking of dangerousness (minimum, medium, high)
  \item The authorities can decide to improve security by highlighting streets in areas at minimum risk
  \item The authorities can decide to improve security by adding cameras in the areas at medium risk 
  \item The authorities can decide to improve security by both adding cameras and doubling patrol shifts  in the areas at large risk
\end{itemize}

\subsection{Maintainability}
\subsubsection{Testing overview}
\begin{itemize}
  \item The system is checked in its correct functioning by software testing
  \item Unit tests, Integration tests and System tests are performed before and after every update
\end{itemize}

\subsubsection{The system is controlled and monitored}
\begin{itemize}
  \item The system is equipped with condition-monitoring algorithms 
  \item The condition-monitoring algorithms identify the functions that cause alarm
  \item A unit test is made as soon as a function is identified as potentially at risk
  \item An integration test is performed after the unit test of the potentially at risk function
  \item A system test is performed after the integration and unit test of the potentially-at-risk function
\end{itemize}

\subsubsection{The code must be clean and easy to understand}
\begin{itemize}
  \item Code must follow common best practices
  \item It is advisable to follow the design patterns to provide a standard terminology and to make the software more adaptable to future extensions and improvements
  \item Complete and detailed documentation, even for low level functions, is mandatory in order to keep the maintainability on the highest level on the whole system
\end{itemize}

\subsection{Portability}
\begin{itemize}
  \item The software shall run on Windows, macOS, Linux, Android and iOS through a modern web browser supporting at least ECMAScript 2015
  \item To guarantee the portability of the front end application the software shall be developed using a web framework such as React
  \item This approach for the front end application will enable code reuse among native mobile platforms (iOS and Android) and the Web Application
  \item The back end software shall be developed in a language that can be compiled to run on virtual runtime such as Java, JavaScript or C\# which can target the JVM, Node.js and dotnet core respectively
  \item The software can be developed using the standard API's that span different types of operating systems
  \item The software will then be able to be transferred with no modifications on other destination machines
  \item The architecture must be flexible
\end{itemize}
