\section{Functional Requirements}

\subsection{R1: Allow a visitor to register on the mobile phone app or the web app}
\begin{itemize}
    \item The system must allow the visitor to provide credentials and personal data
    \item The system must verify the correspondence between the ID number provided by the visitor and their personal information
    \item The system must allow the visitor to verify the account with an e-mail or SMS verification code
    \item The system must send a recap of registration to email of the visitor
    \item The system must verify that there are no other registered users or authorities with the same e-mail or ID number
    \item The user to accept users data privacy conditions to successfully register to the system
\end{itemize}
\subsubsection{Scenario 1}
Mario Rossi saw the advertisement of SafeStreets and decided to download the mobile application in order to be allowed to notify violations and improve safety of streets in his city. After
opening the new app, he is asked to fill a form with all his personal information, full name, email address, mobile number, ID card, and phone number. To proceed with his
registration, after all having filled all the form, Mario clicks on the "Create account" button. The system verify the correspondence between inserted personal data and the ID card information,
and if the verification is positive, Mario is asked whether he prefers to confirm his registration by phone number or by email.
He chooses the alternative he prefers, then he immediately receives the verification code on the account he chose.
He can confirm his registration by clicking on the verification code. The system then sends a recap to notify the positive outcome of registration. Mario is now a Registered
User of SafeStreets; he can login to notify violations on the streets and is enabled to the use of all the functions for a basic user (not an authority) that the system provides.

\subsection{Allow a user to send pictures about street and parking violations}
\begin{itemize}
    \item A user can submit a picture to the system whenever they see a street violation
    \item SafeStreets analyses the image to read the license plait of the vehicle
    \item The user can manually attach the license plait to the image, to help the system with its validation
    \item The user must provide the type of violation, either selecting it from a list or manually typing it if it is not already in the system
    \item The user can help localizing the violation by inserting the name of the street where it occurred
\end{itemize}
\subsubsection{Scenario 2}
Mario Rossi is a user of SafeStreets, he has just finished working, and he wants to go back home, but he is prevented by a car, that is in double-row parking with his own. He waits for ten 
minutes to see if the owner of the car is coming, but then decide to notify the violation. He makes two pictures representing the violation, then logs into the mobile application, 
inserting his personal data, then he clicks on the button to notify a violation, he insert time (17.55), position ( Via Botticelli, Viale Romagna, 2, 20133 Milano MI), a brief description 
of the violation, and the two pictures. He is asked to click the category of the notified violation (Double-Row Parking), or to add it, in the field "other" if the type of violation is not still 
present in the system. He is then asked to insert the license plate of the vehicle involved in the violation in the mobile application, he does it, then the app verifies that it 
corresponds to the license plate identified with the photograph.

\subsection{Validation of reported violations}
\begin{itemize}
    \item Whenever a violation is reported by a user, it is sent to a random group of $k$ SafeStreets users
    \item Users who receive the notification about this approval process are then asked to approve or reject the report or declare neutral if they are not able to verify
    \item The report is validated if and only if the algebraic sum of approvals (+1), neutrality (0) and rejections (-1) is at least $v$
    \item Every user is supposed to respond, the system expects to receive k answers
    \item If the answers of the users for validation, after a settled timeout, are not all received, the answer is automatically considered neutral
\end{itemize}
\subsubsection{Scenario 3}
SafeStreets has received the notification of Mario Rossi regarding a violation of double-row parking. To verify the violation, it sends a request to a 
number k of users. The number of users is randomly chosen, and depend from the number of users logged in the application at the time the system received the report of Mario Rossi
and the users that were logged just a few time before, to increase probability they will be available to answer the report. K users receive the request for validation, and they start 
answering the report. The system receives reports and makes the algebraic sum of approvals and rejections. The calculated sum is equal to $m$ and comparing the values, the result of the 
inequation $m$>$v$ is positive; as a consequence, the system validates the report and registers the violation.

\subsection{Allow a user to see the areas where a violation is more likely to happen}
\begin{itemize}
    \item SafeStreets can aggregate data inputted by users to show relevant statistics about the frequency of street violations
    \item Users can see how many violations usually happen in their neighborhood, in their current location or in an area of their choice
    \item A user has access to the type and amount of violations that happened
\end{itemize}
\subsubsection{Scenario 4}
Mario Rossi searches for a language school to improve English for his daughter, he found different schools in his city on the internet, but he wants an area 
not too far from his house where the little girl can securely go also on foot. He has a list of five schools to decide. He is a user of SafeStreets, so 
for every school in his list, he looks on the web interface of the application for the statistics about car accidents in all the five zones. He insert the 
address of the school, and the system responds with a map, showing the near unsafe area if the school is in a safe area, or, if the school is in an unsafe 
area, showing more detailed data about frequency of car accidents and violations. This operation is repeated five times.

\subsection{Register the violation}
\begin{itemize}
    \item The reported violation must have been validated by the system
    \item The system checks the type of violation
    \item In case the violation is already supported by the system, the number of violations of that type is increased, the license plate of the involved vehicle is added to the list of offenders, and in a report that SafeStreets will deliver to authorities
    \item In case the violation is not supported by the system, the system updates the list of supported violations, then proceed as in the previous case
\end{itemize}
\subsubsection{Scenario 5}
Luca Bianchi is in his lunch break, with his colleagues, and he is waiting for crossing the road on his way to go back to the office. While stopped at the semaphore

\subsection{Allow authorities to see data about violations and who committed them}
\begin{itemize}
    \item Authorities can access all data about violations that a standard user can access
    \item Authorities also have access to more specific data about who committed a violation, like the license plait of the car or how many infractions have been associated to a specific car
\end{itemize}

\subsection{Allow any citizen to become a user by providing a document}
\begin{itemize}
    \item A citizen can register to SafeStreets by providing their ID number
    \item The system must allow a user to register with an email and a password, that will be asked every time they log in
\end{itemize}

\subsection{Allow authorities to register with a special user profile}
\begin{itemize}
    \item Authorities must first register as citizens
    \item A standard user profile can be upgraded to authority profile if it is verified by the system
    \item To obtain privileged access, the user must provide a valid document that proves their authority status
\end{itemize}

\subsection{Unsafe Areas Identification}
\begin{itemize}
  \item SafeStreets can cross information from different sources to identify potentially unsafe areas
  \item SafeStreets may be integrated with a public service, offered by the municipality, that provides such information
  \item Once identified, SafeStreets can suggest possible interventions to prevent accidents
\end{itemize}

\subsection{Traffic Ticket Generation}
\begin{itemize}
  \item SafeStreets can generate tickets for traffic violations reported from registered users
  \item The information is kept safe and intact through the chain of custody, from the end user to the local police officer issuing the tickets
\end{itemize}