\section{Functional Requirements}

\subsection{Allow a user to send pictures about street and parking violations}
\begin{itemize}
    \item A user can submit a picture to the system whenever they see a street violation
    \item SafeStreets analyses the image to read the license plait of the vehicle
    \item The user can manually attach the license plait to the image, to help the system with its validation
    \item The user must provide the type of violation, either selecting it from a list or manually typing it if it is not already in the system
    \item The user can help localizing the violation by inserting the name of the street where it occurred
\end{itemize}

\subsection{Validation of reported violations}
\begin{itemize}
    \item Whenever a violation is reported by a user, it is sent to a random group of $k$ SafeStreets users
    \item Users who receive the notification about this approval process are then asked to approve or reject the report
    \item The report is validated if and only if the algebraic sum of approvals (+1) and rejection (-1) is at least $v$
\end{itemize}

\subsection{Allow a user to see the areas where a violation is more likely to happen}
\begin{itemize}
    \item SafeStreets can aggregate data inputted by users to show relevant statistics about the frequency of street violations
    \item Users can see how many violations usually happen in their neighborhood, in their current location or in an area of their choice
    \item A user has access to the type and amount of violations that happened
\end{itemize}

\subsection{Allow authorities to see data about violations and who committed them}
\begin{itemize}
    \item Authorities can access all data about violations that a standard user can access
    \item Authorities also have access to more specific data about who committed a violation, like the license plait of the car or how many infractions have been associated to a specific car
\end{itemize}

\subsection{Allow any citizen to become a user by providing a document}
\begin{itemize}
    \item A citizen can register to SafeStreets by providing their ID number
    \item The system must allow a user to register with an email and a password, that will be asked every time they log in
\end{itemize}

\subsection{Allow authorities to register with a special user profile}
\begin{itemize}
    \item Authorities must first register as citizens
    \item A standard user profile can be upgraded to authority profile if it is verified by the system
    \item To obtain privileged access, the user must provide a valid document that proves their authority status
\end{itemize}

\subsection{Unsafe Areas Identification}
\begin{itemize}
  \item SafeStreets can cross information from different sources to identify potentially unsafe areas
  \item SafeStreets may be integrated with a public service, offered by the municipality, that provides such information
  \item Once identified, SafeStreets can suggest possible interventions to prevent accidents
\end{itemize}

\subsection{Traffic Ticket Generation}
\begin{itemize}
  \item SafeStreets can generate tickets for traffic violations reported from registered users
  \item The information is kept safe and intact through the chain of custody, from the end user to the local police officer issuing the tickets
\end{itemize}