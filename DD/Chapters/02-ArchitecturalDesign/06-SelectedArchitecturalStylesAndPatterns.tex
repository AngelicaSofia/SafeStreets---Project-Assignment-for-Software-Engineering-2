\section{Selected Architectural Styles and Patterns}

SafeStreets will be built using a RESTful architecture. The web application and the mobile application will interact with the server using
HTTP endpoints communicating using the JSON format.
The main advantages of these choices are:
\begin{itemize}
  \item separation of the presentation layer from the application logic at a physical level: delegating
  all the presentation logic to the client device will reduce the load of the server (no Server Side Rendering)
  \item the JSON format will ensure the compatibility with all the platforms: being a standard communication format
  guarantees the existence of a variety of libraries for all client platforms
\end{itemize}

The server application will be structured using the concept of middlewares.
Middlewares are chainable components in the HTTP Request-Response flow and guarantee a good
level of isolation among the different functionalities they offer, thus providing a simple pattern that
encourages the employment of the Single Responsibility Principle.

The middlewares we'll be using are:
\begin{itemize}
  \item CompressionMiddleware: it reduces the overall bandwidth needed by the server using a compression algorithm that decreases the size of the data sent (and received) by (and to) the server.
  This middleware will be provided by the Reverse Proxy
  \item AuthenticationMiddleware: it has the purpose of filtering requests, rejecting those which lack the permissions to access a specific resource
  \item LoggingMiddleware: it logs the server traffic. This can be useful for debugging purposes and statistical analysis on further improvement of the infrastructure
\end{itemize}

The SafeStreets APIs will be developed using \textbf{express}, a lightweight library for Node.js that provides an easy
programming interface for building middlewares and REST applications.
The main language we'll be using is TypeScript due to its statically-typed nature. Moreover, being it a superset of JavaScript,
it guarantees the compatibility with a large amount of web-oriented libraries and first-class Functions.
Functional programming fits well with the stateless concept of a RESTful architecture, we'll be following a Functional style for our code,
trying to avoid state whenever possible to reduce the possibility of error caused by mutability.

The database system we'll be using is PostgreSQL, a common RDBMS, on which we'll store all kind of data except for multimedia content. This content,
being non-relational and prevalent on our platform will be stored on a Cloud Storage solution (such as Google Cloud Storage) and linked to the RDBMS
using its own identifiers.
