\section{Component View}
In this section, the individual components will be presented in terms of their functionalities, their role and the 
needed sub-parts, as well as how the sub-elements interface one with another within the overlaying component. Moreover, 
in this section it will be specified which of the considered sub-elements are in charge of interfacing with the other 
components of the system.

\subsection{Database}
The database layer must include a DBMS component, in order to manage operations like insertion, update or deletion as 
well as logging of transactions on data inside the storage memory. The DBMS must guarantee the correct functioning of 
concurrent transactions and the ACID properties. As the application doesn’t require a more complex structure than the 
one provided by the relational data structure, the database has to be a RDBMS. The data layer will be exclusively 
accessible through the Application Server via a dedicated interface. Besides, the Application Server must provide a 
persistent unit to handle dynamic behavior of all the contained application data. 
\newline The system also provides an external Cloud Storage in order to memorize the pictures sent by the users that 
notify a violation. There is a dedicated service for this task as the amount of pictures that can be collected can be 
very huge. After saving pictures in the Cloud Storage, there is a unique identifier for every memorized data.
\newline Sensible data, fundamentally passwords, have to be hashed properly and salted before being stored. 
The users will be granted access only after verification of correct and valid credentials.
\newline Pertinence and integrity are granted, as well as isolation in concurrent transactions and atomicity, all ACID properties 
are satisfied.

\subsection{Application Server}
This layer must handle a great part of the application logic, together with the connections with the data layer and the 
multiple ways of accessing the application from different clients. The main feature of the Application Server is the set 
of specific modules of logic, that describe rules regarding violations, their identification, and the following procedures. 
Work-flows for each of the functionalities provided by the application itself are also provided in this component. 
\newline The interface with the data layer must be handled by a dedicated persistent unit, that will be dedicated also to 
the dynamic data access and management, besides the object-relation mapping. In this way the only Application Server is 
granted to have access to the database.
\newline The Application Server must also provide a way to communicate with external systems, by adapting the application 
to already existing external infrastructures. 
\newline The main logic must include:
\begin{itemize}
  \item \textbf{UserModule}: This module will manage all the logic involved with the user account, management, and operations 
  like registration, login, and also profile customization and update, as allowed to authority, that can register 
  with special permission after they are authorized. Furthermore, the module will deal with the generation and provision 
  of user credentials, connected to the optional two-factors authentication, that can be requested by the client.
  \item \textbf{MapModule}: This module contains the logic used to locate violations and users; besides it deals with the 
  definition of the Unsafe Area boundaries.This module must provide useful data to the logic that is at service of the unit 
  that deals with  Authorities, since it can need localization information, in order to perform its task.
  \item \textbf{AuthorityModule}: This module contains all the logic that grant access to an authority to its specific functionality: 
  generation of traffic tickets and access to personal information and sensible data regarding offenders. Besides, 
  the module is also devoted to the fundamental task of keeping the chain of custody of a certain notification with 
  license plate, in order to preserve the integrity of information.
  \item \textbf{ReportModule}: This module provides the logic behind the report of violations, with particular 
  focus on timing restrictions and on the confirmation of the report by other users. This module is also in charge 
  to detect multiple reports of the same violation.
  \item \textbf{LicensePlateModule}: This module includes the logic needed by other components to set the license plate 
  status. It must also be useful as an interface with the external communication with the municipality, as it forwards a 
  certified license plate, after it has been validated under the supervision of an authority.
  \item \textbf{NotificationModule}: This module is used as a gateway from all the modules that need to interact by 
  sending an email to the clients. Its task is managing the logic behind the email notification services.
\end{itemize}

\subsection{Mobile Client Application}
The mobile client must be designed in such a way to make the communication with the Application Server easy and not dependent 
on the implementation on both sides of communication. In order to reach this goal, adequate APIs must be defined and used 
to manage the interaction between the two components. The mobile application UI must be designed following the procedures 
provided by Android and iOS systems.
\newline The mobile application must provide a software module that manages GPS connection of the device and helps with 
the identification of location. This software will provide collected data to the Application Server to be processed. This 
is useful, but not fundamental for the application as the user is able to use SafeStreets also in absence of GPS 
functioning; in this case the location is manually inserted by the user.

